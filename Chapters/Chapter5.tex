\label{Chapter5}

\chapter{Analysing Ethereum ICOs}

\section{Information sources}
\label{informationsources}
\subsection{ICO-Rating}
\label{icoRating}
\subsubsection{Available Data}
\label{icoRatingAvailableData}
\href{https://icorating.com/}{ICO-Rating} contains a list of active ICOs. It rates them relying on they stability and risk on investment. ICORating rates an ICO using three scores:
\begin{itemize}
    \item \textit{\textbf{HypeScore}}: Hype-score shows investor level of interest in the project. The higher the score, the more people may consider the project for future investments. That is a numeric score between 0 and 5;
    \item \textit{\textbf{RiskScore}}: Risk score is aimed at assessing the risk of potentially fraudulent activities. The higher the risk score, the less information there is on the details of the ICO campaign, product development, the team and the documentation, which calls into question the possibility for success of the start-up and the ICO/Token sale. That is a numeric score between 0 and 5;
    \item \textit{\textbf{Investment Rating}}: The metrics of this parameter is divided into 10 levels: \textit{Positive+, Positive, Stable+, Stable, Risky+, Risky, Risky-, Negative, Negative-, Default}. The higher the rate assigned to the project, the better the overall quality of the project’s documentation, and the lower the number of risks for future investors.
\end{itemize}
\subsubsection{Extracted API Methods}
In this subsection we describe the API internal methods that retrieve data listed in \ref{icoRatingAvailableData}. \newline 
In the following table is written that the methods used to retrieve all the data described above take only the token name as input. \newline
This is because ICORating does not have a REST API, or any other API to call, but we had to do a scraping on the HTML page at the address \texttt{icorating.com/ico/<token-name>/}. \newline

In order to retrieve all this data, we created a class called \textbf{ICORatingAPI}, following are the methods in this class.
\begin{center}
\begin{tabular}{| C{2.5cm} | C{4cm} | C{5.5cm} |} \hline
    \textit{Field Name} & \textit{Description} & \textit{Method Signature}\\ \hline 
    \textbf{Hype Score} & ICORating's hype score for the given ICO & \texttt{getHypeScore(icoName)}\\ \hline 
    \textbf{Risk Score} & ICORating's risk score for the given ICO & \texttt{ getRiskScore(icoName)}\\ \hline 
    \textbf{Investment Rating} & ICORating's investment rating for the given ICO & \texttt{getInvestmentRating(icoName)}\\ \hline 
\end{tabular}
\end{center}

\subsection{ICOBench}
\label{icoBench}
\subsubsection{Available Data}
\href{https://icobench.com/}{ICOBench} is a free ICO rating platform and a blockchain community supported by a wide range of experts that provides analytical, legal, and technical insights to the investors.
ICOBench also gives a rating for each Token. Rating is given in combination of:
\begin{itemize}
    \item Their assessment algorithm that uses more than 20 different criteria on which each ICO can earn more than 30 points and
    \item The rating the independent experts give to the ICO following our rating methodology suggestions.
\end{itemize} 

The rating is a float value beetween 0 and 5.0 and is splitted in four subratings, which are combined (with a simple arithmetic average) to retrieve the general ICO rating:
\begin{itemize}
    \item ICO Profile
    \item Team
    \item Vision
    \item Product
\end{itemize}
ICOBench gives an useful \href{https://icobench.com/developers}{REST API} to retrieve all data they have about 1787 ICO from 164 different countries. \newline
Like ICORating in \ref{icoRating}, in order to retrieve all needed data with rest APIs, we have to use  either the token name or the token symbol. We have to do this because ICOBench does not store (or does not make available) the contract address, which we believe is more unambiguous then token name. \newline
In the next subsection we will talk about all the data that we extracted using this API.
\subsubsection{Extracted API Methods}

In order to extract useful data using this API, we have created a specific class, called \textbf{ICOBenchAPI}, following are the methods in this class.
\begin{center}
\begin{tabular}{| C{2cm} | C{4cm} | C{7cm} |} \hline
    \textit{Field Name} & \textit{Description} & \textit{Method Signature}\\ \hline 
    \textbf{ICO Symbol} & ICORating's investment rating for the given ICO & \texttt{getSymbol(icoName)}\\ \hline 
    \textbf{Exchange Details} & Details about the Exchanges that trade this token & \texttt{getExchanges(icoName/icoSymbol)}\\ \hline 
    \textbf{Market Cap} & Actual Market Capitalization of the given ICO &
    \texttt{getMarketCap(icoName/icoSymbol)}\\ \hline
    \textbf{General Rating} & ICOBench's general rating for the given Token &
    \texttt{getGeneralRating(icoName/icoSymbol)
    }\\ \hline
    \textbf{Profile Rating} & ICOBench's profile rating for the given Token &
    \texttt{getProfileRating(icoName/icoSymbol)}\\ \hline 
    \textbf{Team Rating} & ICOBench's team rating for the given Token &
    \texttt{getTeamRating(icoName/icoSymbol)}\\ \hline 
    \textbf{Vision Rating} & ICOBench's vision rating for the given Token ICO &
    \texttt{getVisionRating(icoName/icoSymbol)}\\ \hline 
    \textbf{Product Rating} & ICOBench's product rating for the given Token &
    \texttt{getProductRating(icoName/icoSymbol)}\\ \hline
\end{tabular}
\end{center}

\subsection{TokenWhoIs}
\subsubsection{Available Data}
\href{https://tokenwhois.com/}{TokenWhoIs} is a website that contains a wide list of ICOs, and for eeach of them, it contains information like Market Capitalization, used blockchain, unit price in various currencies. \newline
Like ICOBench in \ref{icoBench} in order to retrieve all needed data with rest APIs, we have to use  either the token name or the token symbol. 
\subsubsection{Extracted API Methods}
In order to extract useful data using this API, we have created a specific class, called \textbf{TokenWhoIsAPI}, following are the methods in this class.
\begin{center}
\begin{tabular}{| C{2cm} | C{4cm} | C{7cm} |} \hline
    \textit{Field Name} & \textit{Description} & \textit{Method Signature}\\ \hline 
    \textbf{Used Blockchain} & Blockchain used by the given token & \texttt{getUsedBlockchain(icoName)}\\ \hline 
    \textbf{Market Cap} & Actual Market Capitalization of the given ICO &
    \texttt{getMarketCap(icoName)}\\ \hline
    \textbf{Token Unit Price (USD)} & Token current unit price (USD) &
    \texttt{getUSDUnitPrice(icoName)
    }\\ \hline
    \textbf{Token Unit Price (ETH)} & Token current unit price (ETH) &
    \texttt{getETHUnitPrice(icoName, icoSymbol)}\\ \hline 
    \textbf{Token Unit Price (BTC)} & Token current unit price (BTC) &
    \texttt{getBTCUnitPrice(icoName)}\\ \hline 
    \textbf{Exchange Names} & Name of Exchanges that trade this token ICO &
    \texttt{getExchangeNames(icoName)}\\ \hline 
    \textbf{Token Supply (USD)} & Token total supply in USD &
    \texttt{getUSDSupply(icoName,icoSymbol)}\\ \hline
\end{tabular}
\end{center}

\subsection{CoinMarketCap}
\href{https://coinmarketcap.com/}{CoinMarketCap} is a website that provides market data about cryptocurrencies and ICO tokens. It provides data like actual and historical market capitalization and actual and historical cryptocurrency unit price in various currencies.
\subsubsection{Available Data}
CoinMarketCap provides an useful REST API, which prvides the following data for each cryptocurrency:
\begin{itemize}
    \item \textit{Name}: Token Name;
    \item \textit{Symbol}: token Symbol;
    \item \textit{Price USD}: Current currency unit price in USD;
    \item \textit{Price BTC}: Current currency unit price in BTC;
    \item \textit{24h volume USD}: Volume of currency transferred in the last 24 hours, in USD;
    \item \textit{Market Cap USD}: Coin Market Capitalization in USD;
    \item \textit{Available Supply}: Coin available supply. 'Available' means 'available to buy';
    \item \textit{Total Supply}: Coin total supply;
    \item \textit{Percent Change 1h}: Change of coin price in the last hour;
    \item \textit{Percent Change 24h}: Change of coin price in the 24 hours;
    \item \textit{percent Change 7d}: Change of coin price in the seven days;
    \item \textit{Last Updated}: Last time these information were updated.
\end{itemize}
\subsubsection{Extracted API Methods}
In order to extract useful data using this API, we have created a specific class, called \textbf{CoinMarketCapAPI}, following are the methods in this class.
\begin{center}
\begin{tabular}{| C{2cm} | C{4cm} | C{7cm} |} \hline
    \textit{Field Name} & \textit{Description} & \textit{Method Signature}\\ \hline 
    \textbf{Market Cap} & Actual Market Capitalization of the given token &
    \texttt{getTokenMarketCap(icoName, icoSymbol)}\\ \hline
    \textbf{Token Unit Price (USD)} & Token current unit price (USD) &
    \texttt{getTokenUSDPrice(icoName, icoSymbol)}\\ \hline
    \textbf{Token Unit Price (BTC)} & Token current unit price (BTC) &
    \texttt{getTokenBTCPrice(icoName, icoSymbol)}\\ \hline 
    \textbf{Exchange Names} & Name of Exchanges that trade this token ICO &
    \texttt{getExchangeNames(icoName)}\\ \hline 
    \textbf{Token Supply (USD)} & Token total supply in USD &
    \texttt{getUSDSupply(icoName,icoSymbol)}\\ \hline
\end{tabular}
\end{center}

\subsection{EtherScan}
\subsubsection{Available Data}
\href{https://etherscan.io/}{EtherScan} is a Block Explorer and Analytics Platform for Ethereum.
It provides a view of all blockchain containing:
\begin{itemize}
    \item Block information such as block number, hash, miner;
    \item Accounts information;
    \item Transaction information such as hash, receiver, sender, amount;
    \item Internal Transaction information such as hash, receiver, sender, amount, father transaction's hash;
    \item Tokens information such as token name, symbol, owners;
\end{itemize}
Etherscan also provides an useful REST API to retrieve all data they have. In the next section we'll describe all data extracted from Etherscan
\subsubsection{Extracted API Methods}
In order to extract useful data using this API, we have created a specific class, called \textbf{EtherScanAPI}, following are the methods in this class.
\begin{center}
\begin{tabular}{| C{2cm} | C{4cm} | C{7cm} |} \hline
    \textit{Field Name} & \textit{Description} & \textit{Method Signature}\\ \hline 
    \textbf{Token total Supply } & Token total supply &
    \texttt{getTotalSupplyByAddress(icoAddress)}\\ \hline
    \textbf{Token balance by adderess} & Given an account address, it returns its token balance &
    \texttt{getTokenAccountBalance(icoAddress, accountAddress)}\\ \hline
\end{tabular}
\end{center}

\subsection{Ethplorer}
\subsubsection{Available Data}
\href{https://ethplorer.io/}{Ethplorer} is a token blobkchain explorer. For each token, it provides information such as:
\begin{itemize}
    \item Basic token information (Name, Symbol, Price, Total Supply);
    \item Basic token contract information (Address, creator's address, balance (ETH), number of transactions);
    \item Token Operations details inside a time interval (Transaction hash, sender, receiver, amount of token transferred);
\end{itemize}
\subsubsection{Extracted API Methods}
In order to extract useful data using this API, we have created a specific class, called \textbf{EthplorerAPI}, following are the methods in this class.


\section{Collection of ICOs data}
In order to store blockchain data combined with the external data retrieved using functions described in section \ref{informationsources}, we chose to use SQL, for the sake of precision, we used PostgreSQL.
In the following section we will explain in details this view, and how we used it.
\subsection{ICO's View}
The view is composed of four SQL tables: 
\begin{outline}
    \1 \textit{Block}: contains all information about blocks retrieved directly from blockchain. In particularly, it contains the following columns:
        \2 \textit{Hash}: the block hash inside blockchain, which is an unambiguous 32-Byte value and is used as table primary key;
        \2 \textit{Number}: represents the block's order inside blockchain;
        \2 \textit{Parent Hash}: contains the hash of the previous block, considered as its father inside blockchain;
        \2 \textit{Timestamp}: the date and time when this block is mined;
        \2 \textit{Miner}: the account that has mined this block;
    \1 \textit{Transaction}: contains all information about transaction contained inside every block, retrieved directly from blockchain. In particularly, it contains the following columns:
        \2 \textit{Hash}: the transaction hash inside blockchain, which is unambiguous like block ones and is used as table primary key;
        \2 \textit{nonce}: the number of transactions made by the sender prior to this one;
        \2 \textit{transaction index}: the number of this transaction inside its block;
        \2 \textit{from}: the transaction sender's address;
        \2 \textit{to}: the transaction receiver's address;
        \2 \textit{value}: value transferred in this transaction, in Wei;
        \2 \textit{creates}: this field is filled i and only if this transaction creates a contract. It contains the address of the newly created contract;
        \2 \textit{gas}: gas provided by the sender;
        \2 \textit{gasprice}: gas price provided by the sender in Wei;
        \2 \textit{blockhash}: the hash of the block that contains this transaction, it is used as the foreign key for the \textit{Block} table.
    \1 \textit{Internal-Transaction}
        \2 \textit{Id}: this field is used as primary key in this table;
        \2 \textit{Parent Tx Hash}: the hash of the father transaction. that is the transaction that generates this internal transaction;
        \2 \textit{Transaction type}: the internal transaction type (a value between call, suicide, create);
        \2 \textit{From}: the internal transaction sender's address;
        \2 \textit{To}: the internal transaction receiver's address;
        \2 \textit{Value}: alue transferred in this transaction, in Wei;
    \1 \textit{ICO}
        \2 \textit{Id}: this field is used as primary key in this table;
        \2 \textit{Token Name}: The token name;
        \2 \textit{Token Symbol}: The token Symbol;
        \2 \textit{Contract Address}: the address of the ERC20-contract that has created this token;
        \2 \textit{Market Cap}: The token market capitalization;
        \2 \textit{Total Supply}: The token total supply;
        \2 \textit{Price USD}: The current token price (USD);
        \2 \textit{Price BTC}: The current token price (BTC);
        \2 \textit{Hype Score}: The token hype score, given by ICORating;
        \2 \textit{Risk Score}: The token risk score, given by ICORating;
        \2 \textit{Investment Rating}: The token investment rating, given by ICORating;
        \2 \textit{Tx Creator Hash}: The address of the transaction that has created the ERC20-contract, which address is in the \textit{Contract Address} field.
\end{outline}

\section{Empirical analysis of ICOs}