\label{Chapter4}

\chapter{Ethereum ICOs}

\section{Initial Coin Offering}
An \textbf{initial coin offering} (ICO) is a controversial means of crowdfunding centered around cryptocurrency \cite{chohan2017initial,teutsch2017interactive} which can be a source of capital for startup companies. 

In an ICO, a quantity of the crowdfunded cryptocurrency is preallocated to investors in the form of \textit{tokens}, in exchange for legal tender or other cryptocurrencies such as \textbf{Bitcoin} or \textbf{Ethereum}. These tokens supposedly become functional units of currency if or when the ICO's funding goal is met and the project launches.

ICOs provide a means by which startups avoid costs of regulatory compliance and intermediaries, such as venture capitalists, bank and stock exchanges while increasing risk for investors. ICOs may fall outside existing regulations or may need to be regulated depending on the nature of the project, or are banned altogether in some jurisdictions, such as China and South Korea

\section{ERC20 Tokens}
The \href{https://github.com/ethereum/EIPs/blob/master/EIPS/eip-20.md}{ERC20} token standard \cite{vogelsteller2015erc} describes the functions and events that an Ethereum token contract has to implement. \newline
Following is an interface contract declaring the required functions and events to meet the ERC20 standard:
\begin{lstlisting}[language=Solidity]
contract ERC20Interface {
    function totalSupply() public constant returns (uint);
    function balanceOf(address tokenOwner) public constant returns (uint balance);
    function allowance(address tokenOwner, address spender) public constant returns (uint remaining);
    function transfer(address to, uint tokens) public returns (bool success);
    function approve(address spender, uint tokens) public returns (bool success);
    function transferFrom(address from, address to, uint tokens) public returns (bool success);

    event Transfer(address indexed from, address indexed to, uint tokens);
    event Approval(address indexed tokenOwner,address indexed spender, uint tokens);
}
\end{lstlisting}
Some of the tokens include further information describing the token contract:
\begin{lstlisting}[language=Solidity]
    string public constant name = "Token Name";
    string public constant symbol = "SYM";
    uint8 public constant decimals = 18;
\end{lstlisting}

\subsection{How a Token Contract Works}
Following is a fragment of a token contract to demonstrate how a token contract maintains the token balance of Ethereum accounts:
\begin{lstlisting}[language=Solidity]
contract TokenContractFragment {

    // Balances for each account
    mapping(address => uint256) balances;

    // Owner of account approves the transfer of an amount to another account
    mapping(address => mapping (address => uint256)) allowed;

    // Get the token balance for account `tokenOwner`
    function balanceOf(address tokenOwner) 
    public constant returns (uint balance) {
        return balances[tokenOwner];
    }

    // Transfer the balance from owner's account to another account
    function transfer(address to, uint tokens) 
    public returns (bool success) {
    balances[msg.sender] = balances[msg.sender].sub(tokens);
    balances[to] = balances[to].add(tokens);
    Transfer(msg.sender, to, tokens);
    return true;
    }

    // Send `tokens` amount of tokens from address `from` to address `to`
    // The transferFrom method is used for a withdraw workflow, allowing contracts to send
    // tokens on your behalf, for example to "deposit" to a contract address and/or to charge
    // fees in sub-currencies; the command should fail unless the _from account has
    // deliberately authorized the sender of the message via some mechanism; we propose
    // these standardized APIs for approval:
    function transferFrom(address from, address to, uint tokens) 
    public returns (bool success) {
        balances[from] = balances[from].sub(tokens);
        allowed[from][msg.sender] = allowed[from][msg.sender].sub(tokens);
        balances[to] = balances[to].add(tokens);
        Transfer(from, to, tokens);
        return true;
    }

    // Allow `spender` to withdraw from your account, multiple times, up to the `tokens` amount.
    // If this function is called again it overwrites the current allowance with _value.
    function approve(address spender, uint tokens) 
    public returns (bool success) {
        allowed[msg.sender][spender] = tokens;
        Approval(msg.sender, spender, tokens);
        return true;
    }
}
\end{lstlisting}

\subsubsection{Token Balance}
For an example, assume that this token contract has two token holders: \newline
\hspace*{4ex} \texttt{0x1111111111111111111111111111111111111111} with a balance of 100 units
\newline
\hspace*{4ex} \texttt{0x2222222222222222222222222222222222222222} with a balance of 200 units
\newline
The token contract's balances data structure will contain the following information:
\newline
\hspace*{4ex} \texttt{balances[0x11111...] = 100}
\newline
\hspace*{4ex} \texttt{balances0x22222...] = 200}
\newline
The \texttt{balanceOf(...)} function will return the following values:
\newline
\hspace*{4ex} \texttt{tokenContract.balanceOf(0x11111...)} will return 100
\newline
\hspace*{4ex} \texttt{tokenContract.balanceOf(0x22222...)} will return 200

\subsubsection{Transfer token balance}
If \texttt{0x11111...} wants to transfer 10 tokens to \texttt{0x22222...}, \texttt{0x11111...} will execute the function:
\begin{lstlisting}[language=Solidity]
    tokenContract.transfer(0x22222..., 10)
\end{lstlisting}
The token contract's \texttt{transfer(...)} function will alter the balances data structure to contain the following information:
\newline
\hspace*{4ex} \texttt{balances[0x11111...] = 90}
\newline
\hspace*{4ex} \texttt{balances[0x22222...] = 210}
\newline
The \texttt{balanceOf(...)} function will now return the following values:
\newline
\hspace*{4ex} \texttt{tokenContract.balanceOf(0x11111...)} will return 90
\newline
\hspace*{4ex} \texttt{tokenContract.balanceOf(0x22222...)} will return 210

\subsubsection{Approve And TransferFrom Token Balance}
If \texttt{0x11111...} wants to authorise \texttt{0x22222...} to transfer some tokens to \texttt{0x22222...}, \texttt{0x11111...} will execute the function:
\begin{lstlisting}[language=Solidity]
    tokenContract.approve(0x22222..., 30)
\end{lstlisting}
The approve data structure will now contain the following information:
\newline
\hspace*{4ex} \texttt{tokenContract.allowed[0x11111...][0x22222...] = 30}
\newline
If\texttt{0x22222...} wants to later transfer some tokens from \texttt{0x11111...} to itself, \texttt{0x22222...} executes the \texttt{transferFrom(...)} function:
\begin{lstlisting}[language=Solidity]
    tokenContract.transferFrom(0x11111..., 0x22222..., 20)
\end{lstlisting}
The balances data structure will be altered to contain the following information:
\newline
\hspace*{4ex} \texttt{balances[0x11111...] = 70}
\newline
\hspace*{4ex} \texttt{balances[0x22222...] = 230}
\newline
And the approve data structure will now contain the following information:
\newline
\hspace*{4ex} \texttt{tokenContract.allowed[0x11111...][0x22222...] = 10}
\newline
0x22222... can still spend 10 tokens from 0x11111....
The \texttt{balanceOf(...)} function will now return the following values:
\newline
\hspace*{4ex} \texttt{tokenContract.balanceOf(0x11111...)} will return 70
\newline
\hspace*{4ex} \texttt{tokenContract.balanceOf(0x22222...)} will return 230


\section{Related works}
This work is based on previous works. We first developed a tool for Ethereum blockchain analytics, improving an existing tool \cite{bartoletti2017generalbit,bartoletti2017generalblock}.
