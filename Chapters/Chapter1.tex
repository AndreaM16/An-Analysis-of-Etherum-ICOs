\label{Chapter1}

\chapter{Introduction}
The last few years have witnessed a steady growth in interest on blockchains, driven by the success of Bitcoin and, more recently, of Ethereum \cite{wood2014ethereum}. This has fostered the research on several aspects of blockchain technologies, from their theoretical foundations — both cryptographic \cite{clark2015research, garay2015bitcoin} and economic \cite{luu2015power, schrijvers2016incentive} — to their
security and privacy \cite{androulaki2013evaluating, bonneau2014mixcoin, gervais2016security, karame2015misbehavior, meiklejohn2013fistful}.

Among the research topics emerging from blockchain technologies, one that has received major interest is the analysis of the data stored in blockchains. Indeed, the two main blockchains contain several gigabytes of data (∼130GB for Bitcoin, ∼300GB for Ethereum), that only in part are related to currency transfers. Developing analytics on these data allows us to obtain several insights, as well as economic indicators that help to predict market trends.

Many works on data analytics have been recently published, addressing anonymity issues, e.g. by de-anonymising users \cite{meiklejohn2013fistful,ober2013structure,reid2013analysis}, clustering transactions \cite{vasek2015there,harrigan2016unreasonable}, or evaluating anonymising services \cite{moser2017anonymous}. Other analyses have addressed criminal activities, e.g. by studying denial-of-service attacks \cite{baqer2016stressing,vasek2014empirical}, ransomware \cite{liao2016behind}, and various financial frauds \cite{moser2013inquiry}. Many statistics on Bitcoin and Ethereum exist, measuring e.g. economic indicators \cite{lischke2016analyzing}, transaction fees \cite{moser2015trends}, the usage of metadata \cite{bartoletti2017analysis}.

